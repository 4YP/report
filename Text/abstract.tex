\pagenumbering{gobble}
\section*{\centering{Acknowledgements}}

    In addition to the guidance and advice offered to me by my supervisor Lionel Tarassenko, I would also like to acknowledge the contributions of Carmelo Velardo and Dario Salvi to this project. They gave me their time and advice as well as agreeing to walk around the Institute of Biomedical Engineering with pink straps on their legs and feet to aid me in data collection. These three outstanding individuals were more than generous and their contributions were invaluable to the success of this project.

\begin{abstract}

    Chronic diseases affect large sections of the general population and incur large costs to the healthcare system, to the tune of \$1.4 trillion per year in the United States. Patients with chronic diseases require regular evaluation through the collection and analysis of data. This project set out to evaluate the information that could be extracted from accelerometer data for people living with chronic diseases. Algorithms were developed for step counting using the accelerometer within a smartphone device and for sleep detection using a wrist-mounted accelerometer. These algorithms derived two metrics of interest to healthcare professionals, the level of activity and sleep quality. Since smartphones and wearables have high penetration in the population and continue to proliferate, healthcare professionals would be able to utilize these algorithms to enable remote data collection. 

    The step-counting algorithm is modular and based on the Windowed Peak Detection design. A ground-truth device was designed to enable rapid data collection for optimization and validation of the step-counting algorithm. A database was created from the data recordings acquired using this ground-truth device. The database was used to optimize the step counting algorithm resulting in a median step counting accuracy of 96.8\% across the entire database. Optimization for specific scenarios was also performed giving high accuracy, up to 99\% in some cases. 

    The sleep-detection algorithm is based on classification with logistic regression. A publicly available database was utilized for the optimization of this algorithm. A filter with a hard-limiter after was used to process the logistic regression output giving results that more closely resemble sleep patterns. This methodology gives a median total time asleep error of 10.9\%. This improves upon previous attempts on the same database by approximately 10\%. 

    The source code for both algorithms will be released along with the step-counting database that was created during the project. The high accuracy achieved by the step counting algorithm means that it is ready to be implemented in a medical context, for example, in the six minute walk test which is a standard procedure for evaluating the capabilities of patients with congestive heart failure. The improvements achieved by the sleep detection algorithm are a step forward and the nature of the algorithm allows for further refinements and improvements.
\end{abstract}
\pagenumbering{roman}
\tableofcontents
\newpage
\pagenumbering{arabic}
\setcounter{page}{1}

