\part{Introduction}

    \chapter{Motivation}

        \section{Pulmonary Disease and Heart Failure}

            One disease of interest is Chronic Obstructive Pulmonary Disease (COPD) is the name for a set of lung conditions that are the source of breathing difficulties. COPD is a relatively common disease for middle-aged or older adults. According to the British Lung Foundation, around 4.5\% of adults aged 45 and older are diagnosed with this condition [CIT]. Often times the affected does not realize that they have the disease, but their breathing problems get gradually worse. The disease cannot be cured or reversed, but treatment may keep it under control so that it does not impact their daily life. Part of this treatment may be what is known as 'pulmonary rehabilitation', a specialised programme of exercise. [CIT]

            Another disease of interest is heart failure. Heart failure is a condition where the heart is unable to pump blood around the body properly, this usually occurs because the heart is too stiff or weak. According to the British Heart Foundation, around half a million people in the UK have been diagnosed with heart failure [CIT]. There are four classes or stages of heart failure, with a higher class signifying higher severity. 

        \section{Six Minute Walk Test}

            The Six Minute Walk Test (6MWT) is a standardized test of functional exercise that involves walking along a flat, straight course for six minutes. The test is self paced and there are no requirements regarding speed or breaks. The 6MWT is often used to track therapy progress and has been proven to be sensitive to common therapies [CIT]. Typically the variable that is measured is the distance walked in the six minutes. However, oxyhaemoglobin saturation and heart rate can be measured too. 

            The 6MWT is often used as part of tracking a patient's recovery or condition for diseases like those listed above. [CIT]

        \section{Smartphones}

            In recent years the increasing penetration of the smartphone has been rapid and widespread. According to the Deloitte Mobile Consumer Survey 2016 [CIT], 91\% of people in the UK aged 18-44 own a smartphone. Additionally, Statista reports that in 2015 50\% of those surveyed aged 55-64 owned a smartphone and 18\% aged 65+ owned a smartphone [CIT]. Clearly, the adoption of the smartphone is reaching saturation among the general populace, particularly the younger generations.

            Parallel to the increasing adoption of smartphones is the increasing capabilities of such devices. For example, the BLU Diamond M Android Smartphone which is available for \textsterling39.99 on AmazonUK [CIT]. This phone has a 1.3 GHz quad-core processor and 512MB RAM [CIT]. Compare this to flagship devices from prior years, for example, the Samsung Galaxy S3 released in May 2012 at a cost of \$599 at launch. This device had a 1.4 Ghz quad-core processor and 1GB RAM [CIT]. The cost to performance ratio for such devices has dropped dramatically in recent years.

            Another important factor to consider is the availability of sensors in these devices. Even such low end devices as the BLU Diamond M has a built in accelerometer and proximity sensor. [CIT] Higher end devices will have additional sensors such as gyroscope, barometer, or a magnetometer. This availability of sensors across the board allows for a unique opportunity for widespread, remote data collection. 

            For example, the 6MWT described above can be performed by the patient using a certified application that counts the steps automatically. This is far easily and less costly than having the patient attend a testing center to record these results. 

    \chapter{Objectives}

        \section{Step Counter}

            In order to track 6MWT performance on a smartphones, a generalized step counting algorithm must be available on those devices. More modern smartphones often incorporate step counters on device, but older devices still lack this functionality. Furthermore, the mechanism of these step counters are not publicly available.

            To overcome these limitations, a generalized algorithm will be developed to count steps for the 6MWT using solely the accelerometer signal from a smartphone. As accelerometer signals are often very noisy, this algorithm must be robust to noise and capable accuracies sufficiently high to enable meaningful analysis of the results. Additionally, this algorithm should be computationally efficient so as to run in realtime on a smartphone device.

            The performance of this algorithm will be validated against the ground truth data collected by a proprietary device, described later in this report. The metric of interest is the total number of steps over a given time period. This report shall not discuss the false positive rate, the rate at which the algorithm identifies a step when there isn't one, or the false negative rate, the rate at which the algorithm does not identify a step when there is one, of the algorithm.

        \section{Sleep Detection}

            The other component necessary to the project is an accurate sleep detector. While it is unlikely that accelerometer driven sleep detectors will reach the classification accuracy achieved by the gold standard, polysomnography, they represent a very cheap alternative that can provide a rough estimate of sleep onset and duration. Polysomnography is a study that is done on a patient, typically to diagnose sleep disorders. The patient undergoing polysomnography has their brain waves, oxygen levels in the blood, heart rate, breathing, eye movement, and leg movement recorded. The data is later analysed by a trained professional, who produces a dataset relating how the patient transitions between sleep stages (deep sleep, REM, awake) during the night.

            The feasibility of such algorithms has been demonstrated with consumer products such as those sold by Fitbit. The algorithm and methodology behind these commercial trackers are unknown, and so the algorithm developed here represents a transparent attempt at classifying sleep with wrist-mounted accelerometers.

            This algorithm seeks to discern the sleep onset time, the time at which the user falls asleep, and wakefulness onset time, the time at which the user wakes up, using the accelerometer signal provided by a wrist mounted smart device such as the Apple Watch.

            In order to develop and validate the algorithm, a database produced by the Embedded Sensing Systems Group at TU Darmstadt was used [CIT]. This database contains both the actigraphy traces and annotated polysomnography for 42 sleep lab patients.

            The performance will also be tested against similar devices such as a Fitbit device [CIT] or the Sleep As Android application [CIT]. 


    \chapter{Literature Review}