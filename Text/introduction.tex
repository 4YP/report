\part{Introduction}

    \chapter{Motivation}

        \section{Pulmonary Disease and Heart Failure}

        \section{Six Minute Walk Test}

    \chapter{Objectives}

        \section{Step Counter}

            In order for patients to accurately measure their performance on the 6MWT, a generalized step counting algorithm must be available on a smartphone device. More modern smartphones often incorporate step counters on device, but older devices still lack this functionality. Furthermore, the mechanism and accuracy of these step counters are not publicly available. 

            To overcome these limitations, a generalized algorithm will be developed to count steps for the 6MWT using solely the accelerometer signal from a smartphone. As accelerometer signals are often very noisy, this algorithm must be robust to noise and capable of accuracies upward of 90\%. Additionally, this algorithm should be performant enough to run in realtime on a smartphone device.

            The performance of this algorithm will be validated against the ground truth data collected by a proprietary device, described later in this report. The quantity of interest is the total number of steps, false positives and false negatives are not considered in this case.

        \section{Sleep Detection}

            The other component necessary to the project is an accurate sleep detector. While it is unlikely that accelerometer driven sleep detectors will reach the accuracies achieved by the gold standard, polysomnography, they represent a very cheap alternative that can provide a rough estimate of sleep onset and duration.

            The feasibility of such algorithms has been demonstrated with consumer products such as those sold by Fitbit. The accuracy of such devices are unknown, and so the algorithm developed here represents a transparent attempt at classifying sleep with wrist mounted accelerometers.

            This algorithm seeks to discern the sleep onset time, the time at which the user falls asleep, and wakeful onset time, the time at which the user wakes up, using the accelerometer signal provided by a wrist mounted smart device such as the Apple Watch. 

            The performance of this algorithm will be compared to polysomnography records provided by [CIT], as well validated against other similar devices such as a Fitbit or smartphone applications like Sleep as Android.


    \chapter{Literature Review}

        \section{Step Counting Algorithms}

            There has been a number of previous authors attempting to develop step counting algorithms using accelerometers; the algorithms can be broadly broken down into the three categories: time-based, frequency-based, and machine learning algorithms. Each of these will be discussed in turn.

            \subsection{Time Based Algorithms}

                Time based algorithms aim to utilize the time-series properties of the accelerometer signal in order to extract step information.

                \subsubsection{Windowed Peak Detection}

                    This algorithm utilizes a smoothing filter to remove noise from the signals, then searches for peaks associated with the heel strikes. [Placeholder for talk about implementation and accuracy here]

                \subsubsection{Mean Crossing Counts}

                    This algorithm also uses a smoothing filter to remove noise from the signal. The signal is then segmented into windows and steps are counted by counting the number of positive crossings of the window means. [Placeholder for talk about implementation and accuracy here]

                \subsubsection{Normalized Autocorrelation}

                    This algorithm uses the periodicity of the signal to compute fractional steps. The autocorrelation is computed for rolling two second windows, the lag associated with the maximum autocorrelation is taken as the step period and hence a fractional step can be calculated for the window by:

                    \begin{equation}
                        f_{step} = T_{window} * L_{window}
                    \end{equation}

                    [Placeholder for talk about implementation and accuracy here.]
            \subsection{Frequency Based Algorithms}

                Frequency based algorithms aim to utilize the frequency content of the accelerometer signals to count steps.

                \subsubsection{Short-Term Fourier Transform}

                    This algorithm uses the Short-Term Fourier Transform to calculate fractional steps for windows. The fractional step is calculated by looking at the dominant frequency and the window size. The Short-Term Fourier Transform is given by:

                    \begin{equation}
                        STFT\{x(t)\}(\tau,\omega) = X(\tau,\omega) = \int_{-\infty}^{\infty}x(t)w(t-\tau)e^{-j\omega t}dt
                    \end{equation}

                    Where $w(t-\tau)$ is the windowing function chosen.

                    [Placeholder for talking about implementation and accuracy here.]

                \subsubsection{Continuous/Discrete Wavelet Transform}

                    This algorithm uses a fusion of frequency and time based methods. The signal is first transformed into the frequency domain using a wavelet transform. All the frequency content outside the 'walking band' is removed, and the remaining signal is transformed back into the time domain.

                    The steps are then counted using the mean crossing technique discussed above, without the smoothing filter.

            \subsection{Machine Learning Algorithms}

                \subsubsection{Hidden Markov Model}

                \subsubsection{K-Means Clustering}

        \section{Sleep Detection Algorithms}